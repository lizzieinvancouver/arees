\documentclass[11pt,letter]{article}
\usepackage[top=1.00in, bottom=1.0in, left=1.1in, right=1.1in]{geometry}
\renewcommand{\baselinestretch}{1}
\usepackage{graphicx}
\usepackage{natbib}
\usepackage{amsmath}
\usepackage{hyperref}
\usepackage{todonotes}


\def\labelitemi{--}
\parindent=0pt

\begin{document}
\bibliographystyle{/Users/Lizzie/Documents/EndnoteRelated/Bibtex/styles/besjournals}
\renewcommand{\refname}{\CHead{}}

\title{Effects of phenology on plant community assembly and structure }
\author{}
\date{\today}
\maketitle
\tableofcontents

\section{Deadlines \& info}
Google doc link: \url{https://docs.google.com/document/d/1hSAbi-9lcc7Pek-CGSs_uZRHI6SD3DPK_fiHJ8Rkz7Y/edit?usp=sharing}\\

Space for a total of 8,500 words and 120 references has been reserved for your review—counts that will produce our desired 25 typeset pages. This word count is meant to include tables and figures. Each moderately sized figure/table is estimated at approximately 300 words; each large one, 600 words.\\

It should emphasize where research in a given area should go, as well as where it has been, such that it will influence the future course of knowledge. 

\begin{itemize}
\item 1-2 page outline: 16 March 2022 % April 11-12th: off jury duty and connect then. Maybe connect 20-24 March 
\item DUE: 17 January 2024
\end{itemize}

\section{Outline}
% Leibhold review: https://www.annualreviews.org/doi/pdf/10.1146/annurev-ecolsys-102220-024934
% Marc Mangel: https://users.soe.ucsc.edu/~msmangel/art_bookchap1.html 
{\bf \Large Do next!} Figure out the stuff at the bottom of the figures section (what was I doing? Where does it go or is it just a list to keep ...?). Then, go back to google doc and see where left out stuff fits in (start at `What do we want people to after reading this?'). 

\begin{enumerate}
\item Intro \todo[inline]{Elsa; see 1. in Google docs}
\item Quick overview of assembly theory \todo[inline]{Lizzie; see 2. in Google docs}
\item Where phenology fits in the environmental filtering part of community assembly (subheader?)
\begin{enumerate}
\item Species can only pass environmental filter if they can reproduce within length of growing season (PHENOFIT -- predicts species range limits); old work on fruit size etc.
\item Dynamics of resources across the season may also functionally filter species through their phenology (left side of resource figure)
\begin{enumerate}
\item Simplest model is chemostat: species pass filter if levels of resource are high enough
\item Evaporating single pulse resource: species may invade only at certain levels of resource (includes snowpack/soil nutrients etc.)
\item Multiple pulses: some species may persist through whole season or use first or second pulse only
\end{enumerate}
\end{enumerate}
\item Before we dive into biotic interactions: constraints on phenology x other traits (subheader?) \todo[inline]{Elsa; see google docs}
\item Limiting similarity of phenology  (subheader?) % Part of Biotic interactions
\begin{enumerate}
\item Do species with similar phenologies actually compete? (Temporal niches etc.)
\item Dynamics of resource availability across the season and creates temporal niches (right side of resource figure, see notes on that below in figure section) % Careful here though as fluctuating resources do not provide space for coexistence
\end{enumerate}
\item Priority effects % Part of Biotic interactions
\begin{enumerate}
\item All temporal niches may not be created equally however, because of priority effects 
\item Priority effects and assembly theory – competitive exclusion is not even across the growing season % not 100% sure what we meant here...
\item Seasonal priority effects \emph{are} phenology. (Fukami, Stuble). Review them.
\item Priority effects suggest there should be a drive to be early, which we do see in some data (flowering times etc.) % Blackford?
\item But they have costs: herbivory apparency, frost risk etc. 
\item And, priority effects are not always competitive … Phenological facilitation (Lindsay Leverett 2017: Germination phenology determines the propensity … and old work from the 1980s that Lizzie cannot remember the name of but Dan B must know and other literature (seedlings die))
\todo[inline]{Could we fit applications into this section?}
\end{enumerate}
\item Phenological coexistence
\begin{enumerate}
\item Current landscape of phenological coexistence theory/experiments
\begin{enumerate}
\item `Modern coexistence' -- stabilizing/equalizing 
\item Godoy (and others, Blackford etc.) parameterize models to show trade-offs with phenology % Blackford et al manipulated germination time & found both species had an advantage of germinating early. If the worse competitor germinated early it could outcompete the better competitor. They parameterized a coexistence model. https://www.journals.uchicago.edu/doi/full/10.1086/708719
\item Will review which models have been used for parameterization -- are they all the same?
\item And maybe also compare the types of experiments: All focused on annuals in  which systems ...
\item Highlight limitations % So, there’s no real storage effect in these models? (asks Lizzie)
\end{enumerate}
\item Future potential for phenological coexistence theory/experiments
\begin{enumerate}
\item Exciting time for coexistence theory as new issues arise (Barabas, Song papers)
\item Phenology could help push theory forward...
\item Beyond annual plants 
\item Germination leads to other events ... Community assembly is all about germination/growth and assumes species will flower and set seed (but most studies in modern coexistence only measure seed set, so…)
\item Connect here to \emph{Arabidopsis} models (and common garden across Europe) which is about germination, flowering and seed set (spins back up to life history theory) ... do we need a cross-continental phenological coexistence experiment to (highlight limitations and) push field forward? 
\item Maybe also connect to Chuine... Process-based models focuses on costs of being too early (priority effects?) and whether you can grow in time
\end{enumerate}
\end{enumerate}
\item One step ahead, one step behind % Catch that magic moment 
\begin{enumerate}
\item Phenology -- runs straight at the tension between life history theory and coexistence theory as somehow separate % Are they separate? I am lost. 
\item Maybe something about evolution and community assembly theory here? % https://www.annualreviews.org/doi/pdf/10.1146/annurev-ecolsys-102220-024934
\item What cues drive priority effects (and thus can we predict them)?
\item Maybe something about annual/perennial divide ... which is also a systems divide: Phenology is so focused on temperate deciduous forests and coexistence theory is drought annual systems 
\item Bet-hedging, it's a bad romance. 
\end{enumerate}
\end{enumerate}

{\bf New refs}\\
Asymmetric competition between plant species (Connolly \& Wayne) % https://esajournals.onlinelibrary.wiley.com/doi/10.1890/0012-9658%282001%29082%5B2696%3ARSEDFI%5D2.0.CO%3B2

\section{Figures}

\begin{enumerate}
\item Competition versus phenological overlap -- Can we find real data for this?
\item Resource pulses (temporal resource supply) and coexistence models (chemostat to mid-season drought)  -- Lizzie
\begin{enumerate}
\item Left panel is just Resource x intra-annual time graphs (referenced in filtering section); Right panel is outcome of niches from coexistence theory 
\item Chemostat model – no resource variation so no temporal assembly, just Tilman's R$^*$
\item Evaporating single pulse resource (includes snowpack/soil nutrients etc.)– competition colonization trade-off (lottery model and eventually storage effect) 
\item Multiple pusles (monsoon systems of Venable; mid-season drought systems) -- can we get real data from NEON perhaps?
\end{enumerate}
\item Multi-dimensional trait space … early and risky versus competitive and late – list traits … could do along a season or do a multi-panel figure covering classic frameworks and where phenology fits in:
\begin{enumerate}
\item Growth x defense
\item Grime triangle
\item Competition x colonization
\end{enumerate}
\item Evidence for trade-offs based on data (competition - colonizer, where colonizer = growth rate trade-off)
\item Maybe … Bet-hedging and speed of germination trade-off (speed of germination = germination cues) …. How much you germinate and how fast you germinate are related… Or is this just showing early is fast?
\end{enumerate}


What is the current role of coexistence in community assembly theory (a la phenology)?

Advances in other systems that could be applied to plant systems? (Daphnia resting stages; amphibians) 

Think about how much we'll get into invasions/applications before applications section. 
% Dryad -- is it UC? 
Variance partitioning -- how much does a difference 

Getting ahead of your competitors (biotic component) of phenology may be less important than getting it right abiotically.

Relative importance of phenology relative to other traits ... 

Relative importance of abiotic versus biotic as a section ... 

Schmitt -- relative importance of different phenophases

Intensity vs. importance of competition debate: \\
Intensity -- amount that competition reduces biomass\\
Importance -- relative role of competition. 

\section{Miscellaneous}

\end{document}

\begin{enumerate}
\item 
\end{enumerate}
