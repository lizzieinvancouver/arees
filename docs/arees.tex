\documentclass[11pt]{article}
\usepackage[top=1.00in, bottom=1.0in, left=1.1in, right=1.1in]{geometry}
\renewcommand{\baselinestretch}{1}
\usepackage{graphicx}
\usepackage{natbib}
\usepackage{amsmath}
\usepackage{hyperref}
\usepackage{todonotes}


\def\labelitemi{--}
\parindent=0pt

\begin{document}
\bibliographystyle{/Users/Lizzie/Documents/EndnoteRelated/Bibtex/styles/besjournals}

\title{Effects of phenology on plant community assembly and structure }
\author{}
\date{\today}
\maketitle
\tableofcontents

\section{Deadlines \& info}
\href{https://docs.google.com/document/d/1hSAbi-9lcc7Pek-CGSs_uZRHI6SD3DPK_fiHJ8Rkz7Y/edit?usp=sharing}{Google doc link to original very drafty outline}. {\bf Lizzie moved over all text} except for `Stuff we’re working on …' 14 March 2023. \\

Space for a total of 8,500 words and 120 references has been reserved for your review—counts that will produce our desired 25 typeset pages. This word count is meant to include tables and figures. Each moderately sized figure/table is estimated at approximately 300 words; each large one, 600 words.\\

It should emphasize where research in a given area should go, as well as where it has been, such that it will influence the future course of knowledge. \\

\emph{Original from Kathleen Donahue: }\\
There is great interest in the role of phenology within the context of climate change, including how it may alter species interactions. However, there is much less synthesis on how phenology might influence processes of community assembly, including priority effects, the balance of competition to facilitation, and ultimately species coexistence and community composition. Different phenological transitions (e.g. germination, budburst, reproduction) are likely to have different effects.


\begin{itemize}
% \item 1-2 page outline: 16 March 2022 % April 11-12th: Elsa off jury duty and connect then. Maybe connect 20-24 March 
\item DUE: 17 January 2024
\end{itemize}

\section{Outline}
% Leibhold review: https://www.annualreviews.org/doi/pdf/10.1146/annurev-ecolsys-102220-024934
% Marc Mangel: https://users.soe.ucsc.edu/~msmangel/art_bookchap1.html 
\begin{enumerate}
\item Introduction % Elsa
\begin{enumerate}
\item Importance of phenology  
\begin{enumerate}
\item Define phenology: timing of important growth and reproductive events and the transitions between them
\item Phenological traits determine the experienced biotic and biotic environment; hence, phenology is often related to fitness
\item Speaking of fitness, shifts in phenology have been repeatedly linked to shifts in fitness and growth outcomes with climate change \citep{Cleland:2012}
\end{enumerate}
\item Phenology is one trait embedded within the many traits that determine the suitability of an organism for its environment but is missing from many global analyses of functional trait frameworks.
\item Here we: Examine phenology within major theories of community assembly and life-history trade-offs
\item We will not cover these topics, which have been the subject of other reviews % As they are very well-trodden
\begin{enumerate}
\item Phenology shifts with climate change \citep{menzel2020}
\item What cues underlie phenology (might not know the answers, but well discussed) \citep{chuinearees}
\item Trophic mismatch \citep{kharouba2018}
\end{enumerate}
\end{enumerate}
\item Quick overview of assembly theory %Lizzie
\begin{enumerate}
\item Community assembly
\begin{enumerate}
\item Regional species pool
\item Environmental filtering
\item Biotic interactions (competitive/facilitative; priority effects)
\item Boom! Communities
\end{enumerate}
\item Two big places where phenology matters (that we cover next)…
\begin{enumerate}
\item Filtering
\item Biotic interactions
\end{enumerate}
\end{enumerate}
\item Where phenology fits in the environmental filtering part of community assembly (subheader?)
\begin{enumerate}
\item Species can only pass environmental filter if they can reproduce within length of growing season \citep[PhenoFit model predicts species range limits, see][]{Chuine:2001ab}; fruit size etc.
\item Dynamics of resources across the season may also functionally filter species through their phenology (left side of resource figure)
\begin{enumerate}
\item Simplest model is chemostat: species pass filter if levels of resource are high enough
\item Evaporating single pulse resource: species may invade only at certain levels of resource (includes snowpack/soil nutrients etc.)
\item Multiple pulses: some species may persist through whole season or use first or second pulse only
\end{enumerate}
\end{enumerate}
\item Phenology as a key trait that is under-included in trait theory (subheader?) % Before we dive into biotic interactions: constraints on phenology x other traits 
\begin{enumerate}
\item Constraints from life history theory including life history trade-offs that constrain trait combinations and prevent the Darwinian demon
\begin{enumerate}
\item Growth-defense trade offs, within a season, fast-growing plants are less likely to be defended, incurring a cost to early-activity \cite{waterton2016,meineke2019}
\item Fast growth – competitive trade-off: early-species grow fast but poor resource competitors; akin to Competition-colonization trade-off, good colonizers are fast-growing % (Bin et al. 2019 for subtropical trees) % Need to see if we can find evidence for this. 
\item Within versus across year trade-offs \citep{silvertown1981,Wilczek:2009oa}
\item Partitioning season for vegetative versus fruit production (seed production)
\end{enumerate}
\item Therefore, when considering phenology within community assembly, we need to remember that it is part of a non-random trait framework
\end{enumerate}
\item Limiting similarity of phenology % Part of Biotic interactions
\begin{enumerate}
\item Do species with similar phenologies actually compete? (Temporal niches etc.)
\item Dynamics of resource availability across the season and creates temporal niches (right side of resource figure, see notes on that below in figure section) % Careful here though as fluctuating resources do not provide space for coexistence
\end{enumerate}
\item Advancing trait ecology through phenology % Idea here is to wrap up traits part of the paper mostly and move to more coexistence and assembly theory
\begin{enumerate}
\item Variance partitioning -- how much does a shift in phenology compare to other shifts in life history traits?
\begin{enumerate}
\item Relative importance of abiotic versus biotic as a section
\item Intensity vs. importance of competition % Intensity -- amount that competition reduces biomass
\item Getting ahead of your competitors (biotic component) of phenology may be less important than getting it right abiotically.
\end{enumerate}
\item Phenology as response trait:  to use it usefully must decompose into environmental responses (e.g., chilling or strat etc.), but other traits could do this too % combined with response surfaces -- https://esajournals.onlinelibrary.wiley.com/doi/10.1890/0012-9658%282001%29082%5B2696%3ARSEDFI%5D2.0.CO%3B2
\end{enumerate}
\item Priority effects % Part of Biotic interactions
\begin{enumerate}
\item All temporal niches may not be created equally however, because of priority effects 
\item Priority effects and assembly theory – competitive exclusion is not even across the growing season % not 100% sure what we meant here...
\item Seasonal priority effects \emph{are} phenology \citep{connolly1999,fukami2015} % also Stuble
\item Priority effects suggest there should be a drive to be early, which we do see in some data (flowering times etc.) % Blackford?
\item But they have costs: herbivory apparency, frost risk etc. 
\item And, priority effects are not always competitive … Phenological facilitation \cite{leverett2017}
% \todo[inline]{Could we fit applications into this section?}
\end{enumerate}
\item Phenological coexistence
\begin{enumerate}
\item Current landscape of phenological coexistence theory/experiments
\begin{enumerate}
\item `Modern coexistence' -- stabilizing/equalizing 
\item Godoy (and others, Blackford etc.) parameterize models to show trade-offs with phenology % Blackford et al manipulated germination time & found both species had an advantage of germinating early. If the worse competitor germinated early it could outcompete the better competitor. They parameterized a coexistence model. https://www.journals.uchicago.edu/doi/full/10.1086/708719
\item Will review which models have been used for parameterization -- are they all the same?
\item And maybe also compare the types of experiments: All focused on annuals in  which systems ...
\item Highlight limitations, and what's covered well (what we've learned) % So, there’s no real storage effect in these models? (asks Lizzie)
\end{enumerate}
\item Future potential for phenological coexistence theory/experiments
\begin{enumerate}
\item Exciting time for coexistence theory as new issues arise \citep{Barabas2018,song2019}
% Lizzie wants fewer trumpet plots but maybe this is not the place to mention that.
\item Phenology could help push theory forward...
\item Beyond annual plants 
\item Germination leads to other events ... Community assembly is all about germination/growth and assumes species will flower and set seed (but most studies in modern coexistence only measure seed set, so…)
\item Connect here to \emph{Arabidopsis} models, including common garden across Europe \citep{Stinchcombe:2004ec,arabid2011}, which is about germination, flowering and seed set (spins back up to life history theory) ... do we need a cross-continental phenological coexistence experiment to (highlight limitations and) push field forward? 
\item Maybe also connect to Chuine... Process-based models focuses on costs of being too early (priority effects?) and whether you can grow in time
\end{enumerate}
\end{enumerate}
\item One step ahead, one step behind: Phenology as cross-cutting (both a challenge and benefit) % Catch that magic moment 
\begin{enumerate}
\item Bridging from physiological cues to assembly: What cues drive priority effects (and thus can we predict them)?
\item Phenology runs straight at the tension between life history theory and coexistence theory, which are often treated as somehow separate % Are they separate? I am lost. 
\begin{enumerate}
\item Costs in community assembly models % Or more life-history
\item Can phenology help crack the annual/perennial divide? 
\item Which is also a systems divide: Phenology is so focused on temperate deciduous forests and coexistence theory is drought annual systems 
\item Bet-hedging, it's a bad romance. 
\item Transition briefly in to evolution and community assembly theory here % https://www.annualreviews.org/doi/pdf/10.1146/annurev-ecolsys-102220-024934
\item What phenological strategies are selected on?
\end{enumerate}
\end{enumerate}
\end{enumerate}


% https://esajournals.onlinelibrary.wiley.com/doi/10.1890/0012-9658%282001%29082%5B2696%3ARSEDFI%5D2.0.CO%3B2


{\bf What do we want people to after reading this?} 
\begin{enumerate}
\item Phenology people should recognize phenology as one of many traits
\item Trait people should recognize that phenological strategies are product of important trade-offs within life history
\item And what should community assembly theory take away? (Can decide after lit review of current models, but some ideas below)
\begin{enumerate}
\item Current implementation of modern coexistence theory has problems and we can solve them % works only for annuals, no storage effect
\item Cues within assembly models? Or costs within coexistence models?... not sure yet. 
\item Need more life history within community assembly models? More bet-hedging… 
\end{enumerate}
\end{enumerate}

\section{Figures}
\begin{enumerate}
\item Competition versus phenological overlap -- Can we find real data for this?
\item Resource pulses (temporal resource supply) and coexistence models (chemostat to mid-season drought)  -- Lizzie
\begin{enumerate}
\item Left panel is just Resource x intra-annual time graphs (referenced in filtering section); Right panel is outcome of niches from coexistence theory 
\item Chemostat model – no resource variation so no temporal assembly, just Tilman's R$^*$
\item Evaporating single pulse resource (includes snowpack/soil nutrients etc.)– competition colonization trade-off (lottery model and eventually storage effect) 
\item Multiple pusles (monsoon systems of Venable; mid-season drought systems) -- can we get real data from NEON perhaps?
\end{enumerate}
\item Multi-dimensional trait space … early and risky versus competitive and late – list traits … could do along a season or do a multi-panel figure covering classic frameworks and where phenology fits in:
\begin{enumerate}
\item Growth x defense
\item Grime triangle
\item Competition x colonization
\end{enumerate}
\item Evidence for trade-offs based on data (competition - colonizer, where colonizer = growth rate trade-off)
\item Maybe … Bet-hedging and speed of germination trade-off (speed of germination = germination cues) …. How much you germinate and how fast you germinate are related… Or is this just showing early is fast?
\item A box on advances in other systems that could be applied to plant systems? (\emph{Daphnia} resting stages; amphibians)
\end{enumerate}

\section{To do, including do next...}

{\bf \large Do next!} 

\begin{enumerate}
\item Chat with Janneke about what is assembly vs. coexistence and life history versus assembly
\item Start on lit review of phenology coexistence models. Figure out search terms  and table .... We want to know: 
\begin{enumerate}
\item Basics of model formulation so we can figure if they are all the same or, if not, how different
\item Do models include biotic or abiotic costs?
\item Do the models include storage effect?
\item Data info ....
\begin{enumerate}
\item From where?
\item For which species?
\item How many years?
\item Drought or other environmental limitations?
\item Did they measure costs of being early or too late (fruit/seed set)?
\end{enumerate}
\end{enumerate}
\item Work on conceptual figures 
\item Don't forget about: Try to fit in...' under To do list. 
\end{enumerate}

\vspace{2ex}
{\bf \large More to do items} 

\begin{enumerate}
\item Answer: What is the current role of coexistence in community assembly theory (a la phenology)?
\item Go through refs in our 2011 paper, there are some oldies but goodies there perhaps?
\item Try to fit in... 
\begin{enumerate}
\item Is the timing of woody species germination less critical than for herbaceous species?
\item Does dormancy (or cues that underlie dormancy) make you a weaker tracker?
\item Species vs. population level differences (population-level differences bigger in flowering compared to germination) … Margie’s work shows best to germinate fast for everything, unless southern California and then it’s bet-hedging → fit this section in with trait-trade-offs
\item Resource pulses in assembly theory and linking to phenological cues (maybe)
\item Equalizing versus stabilizing mechanisms with climate change (Lizzie’s stuff, Margie’s stuff)
\item Assembly models as including species that evolution has already selected on, while evolutionary models focus on this … and process-models include the costs (Phenofit)
\item Storage effect model: Time for a new world of coexistence?
\begin{enumerate}
\item Kraft et al. 2015 PNAS the best competitors are late active, but that was under one set of environmental conditions, but it doesn’t take trade-offs into account. Under another set of conditions the outcomes could be different. Early phenology must be advantageous under some sets of conditions or it wouldn’t persist in the environment. ....So, there’s no real storage effect in these models? (asks Lizzie)
\item Darwinian demons: annuals with no seed bank, why no late season species?
\item Transient and trending environments – we need theory on this
\item Eco-evolutionary theory (where the evolution matters … could sneak in population)
\item Stabilizing and equalizing mechanisms – repeatable trumpet plots (trumpet plots with control species or treatments)
\end{enumerate}
\end{enumerate}
\item Refs to work in ...
\begin{enumerate}
\item Young TP, KL Stuble, JA Balachowski, CM Werner (2017) Using priority effects to manipulate competitive hierarchies in restoration. Restoration Ecology 25: 114-123.
\item Stuble KL, L Souza (2016) Priority effects: natives, but not exotics, pay to arrive late. Journal of Ecology 104(4): 987-993.
\item Anderson \& Wadgymar: Climate change disrupts local adaptation and favours upslope migration % https://onlinelibrary.wiley.com/doi/full/10.1111/ele.13427
\end{enumerate}
\end{enumerate}




\bibliography{/Users/Lizzie/Documents/EndnoteRelated/Bibtex/LizzieMainMinimal.bib}


\end{document}

\begin{enumerate}
\item 
\end{enumerate}
