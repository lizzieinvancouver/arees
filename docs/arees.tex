\documentclass[11pt,letter]{article}
\usepackage[top=1.00in, bottom=1.0in, left=1.1in, right=1.1in]{geometry}
\renewcommand{\baselinestretch}{1.1}
\usepackage{graphicx}
\usepackage{natbib}
\usepackage{amsmath}
\usepackage{hyperref}
\usepackage{todonotes}


\def\labelitemi{--}
\parindent=0pt

\begin{document}
\bibliographystyle{/Users/Lizzie/Documents/EndnoteRelated/Bibtex/styles/besjournals}
\renewcommand{\refname}{\CHead{}}



{\Large Effects of phenology on plant community assembly and structure }

\section{Deadlines \& info}
Google doc link: \url{https://docs.google.com/document/d/1hSAbi-9lcc7Pek-CGSs_uZRHI6SD3DPK_fiHJ8Rkz7Y/edit?usp=sharing}\\

Space for a total of 8,500 words and 120 references has been reserved for your review—counts that will produce our desired 25 typeset pages. This word count is meant to include tables and figures. Each moderately sized figure/table is estimated at approximately 300 words; each large one, 600 words.\\

It should emphasize where research in a given area should go, as well as where it has been, such that it will influence the future course of knowledge. 

\begin{itemize}
\item 1-2 page outline: 16 March 2022:
\item DUE: 17 January 2024
\end{itemize}

\section{Outline}
% Leibhold review: https://www.annualreviews.org/doi/pdf/10.1146/annurev-ecolsys-102220-024934
% Marc Mangel: https://users.soe.ucsc.edu/~msmangel/art_bookchap1.html

\begin{enumerate}
\item Intro \todo[inline]{Elsa; see 1. in Google docs}
\item Quick overview of assembly theory \todo[inline]{Lizzie; see 2. in Google docs}
\item Where phenology fits in the environmental filtering part of community assembly (subheader?)
\begin{enumerate}
\item Species can only pass environmental filter if they can reproduce within length of growing season (PHENOFIT -- predicts species range limits); old work on fruit size etc.
\item Dynamics of resources across the season may also functionally filter species through their phenology (left side of resource figure)
\begin{enumerate}
\item Simplest model is chemostat: species pass filter if levels of resource are high enough
\item Evaporating single pulse resource: species may invade only at certain levels of resource (includes snowpack/soil nutrients etc.)
\item Multiple pulses: some species may persist through whole season or use first or second pulse only
\end{enumerate}
\end{enumerate}
\item Before we dive into biotic interactions: constraints on phenology x other traits (subheader?) \todo[inline]{Elsa; see google docs}
\item Biotic interactions: Limiting similarity and phenology  (subheader?)
\begin{enumerate}
\item Do species with similar phenologies actually compete? (Temporal niches etc.)
\item Dynamics of resource availability across the season and creates temporal niches (right side of resource figure, see notes on that below in figure section)
\end{enumerate}
\item Biotic interactions: Priority effects \todo[inline]{Stopped here for the day ... Lizzie needs to finish this part of the outline working off Google doc}
\end{enumerate}
https://users.soe.ucsc.edu/~msmangel/art_bookchap1.html


\section{Figures}

\begin{enumerate}
\item Competition versus phenological overlap -- Can we find real data for this?
\item Resource pulses (temporal resource supply) and coexistence models (chemostat to mid-season drought)  -- Lizzie
\begin{enumerate}
\item Left panel is just Resource x intra-annual time graphs (referenced in filtering section); Right panel is outcome of niches from coexistence theory 
\item Chemostat model – no resource variation so no temporal assembly, just Tilman's R$^*$
\item Evaporating single pulse resource (includes snowpack/soil nutrients etc.)– competition colonization trade-off (lottery model and eventually storage effect) 
\item Multiple pusles (monsoon systems of Venable; mid-season drought systems) -- can we get real data from NEON perhaps?
\end{enumerate}
\item Multi-dimensional trait space … early and risky versus competitive and late – list traits … could do along a season or do a multi-panel figure covering classic frameworks and where phenology fits in:
\begin{enumerate}
\item Growth x defense
\item Grime triangle
\item Competition x colonization
\end{enumerate}
\item Evidence for trade-offs based on data (competition - colonizer, where colonizer = growth rate trade-off)
\item Maybe … Bet-hedging and speed of germination trade-off (speed of germination = germination cues) …. How much you germinate and how fast you germinate are related… Or is this just showing early is fast?
\end{enumerate}



\section{Miscellaneous}

\end{document}

\begin{enumerate}
\item 
\end{enumerate}
