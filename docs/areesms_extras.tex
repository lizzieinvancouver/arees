Current coexistence models integrating phenology focus almost entirely on phenology as it relates to resource use and growth. While this makes sense from the lens of coexistence theory, which has long been focused on competition (citeMcPeek), it ignores reproductive events, which may fundamentally constrain growth events. 

These combined constraints create a geometric reality to phenology that is generally ignored in most phenology research today. 

... in a way that is constrained, complex, and rarely discussed in most phenological research today.  

The need to grow and reproduce within these bounds ultimately underlies all phenology, but the complexity of the sequence---including the series of events and how each relates to a potential suite of trade-offs---is rarely discussed. 

Within coexistence and more generally in phenological research, much work focuses on one event---erasing the complexity of phenological events that are all generally dependent upon one another. This is perhaps not surprising, as making predictions for sequences of events requires bridging from coexistence to life history theory, which has worked to predict the optimal schedule of growth and reproductive timing across an organism's lifetime. But recent work showing that differences in life history can lead to long-term persistence of species in an otherwise neutral model (where species are effectively equalized, CITE) suggests the lines between coexistence theory and life history theory perhaps could be usefully blurred. 


\subsection*{Future directions}

... how relevant this phenology x competitive ability  trade-off is in natural systems is debatable. 
 
 Advancing our understanding of how phenology fits within coexistence will likely involve a more head-on reckoning with the challenge of time in community assembly models ... STIFF equations here? 

% -ignoring an additionally suite of likely relevant mechanisms that are fluctuation dependent: relative non-linearity of competition and the storage effect. ... seem highly relevant to phenology and could yield new insights. 

Relevant fields overly focused on different plant life histories: coexistence focused on annuals and storage through seedbanks, while much phenology work focuses on trees and their recurring annual events. 

% Refs I need:
John Park
Law 1979
McPeek book


% Sent to Elsa on last round of review (20 March 2024):
First, we do define the storage effect when we first use it (line 323) and added that in on revision, but I can see it is not adding much so I suggest we change this sentence slightly to remove the two short definitions we tried to add. It would change just the end of the sentence to now read:

These current studies, like much of the work on modern coexistence theory focus on resource partitioning---a fluctuation-independent mechanism of coexistence---ignoring an additional suite of likely relevant mechanisms that are fluctuation dependent: relative non-linearity and the `storage' effect (defined below).

Next, we explain the storage effect in a good amount of detail just a few sentences later in a full paragraph (starts 'Similarly, the underlying mechanisms of storage effect...'). I thought that maybe adding a sentence like this somewhere in the paragraph would help, but I could not figure where:
While some methods of this `storage' relate to storing  resources, the critical attribute of all the methods is that they allow a species population to build up in size in years that species has a fitness advantage, and the resulting larger population size helps to buffer the species in years where it has a fitness disadvantage.

Here's what we have NOW:

These models also generally insert phenology as a coexistence mechanism mostly independent of environmental variation---even though phenology itself varies year to year with environmental variation. These current studies, like much of the work on modern coexistence theory focus on resource partitioning---a fluctuation-independent mechanism of coexistence---ignoring an additional suite of likely relevant mechanisms that are fluctuation dependent: relative non-linearity and the `storage' effect (defined below). While not yet tackled (to our knowledge) and certainly more complex to model and study, these two mechanisms seem highly relevant to phenology. Relative non-linearity promotes coexistence through variation in competitive intensity over time or space, given that species have different nonlinear responses to competition \citep{CHESSON:1994vn,Chesson:2000vd}. Recent work suggests relative non-linearity may be an important and under-appreciated mechanism in plant communities, but attempted to `control' for phenology rather than consider it \citep{hallett2019rainfall}. Yet species' varying phenologies could produce varying competitive intensity over time and/or create the required non-linear response for coexistence via relative non-linearity. 

Similarly, the underlying mechanisms of storage effect---where species vary in their response to the environment, and that response covaries with competition---could clearly relate to phenology. In this model species experience favorable and unfavorable environmental conditions (generally across periods of time), which must coincide with shifts in high intra-specific (favorable) and inter-specific (unfavorable) competition. Species must also have a way to buffer their population growth through unfavorable periods, which is where the term `storage' come from. `Storage' here is a conceptual term that can refer to many diverse mechanisms species use to `store' favorable environmental periods long enough to survive unfavorable periods, which must coincide with limited competition \citep{Chesson:2000vd}. Phenology is often theoretically proposed as a mechanism by which plants could `store' environmentally favorable periods \citep{Chesson:1993gi,Chesson:2004eo}, but rarely tested. Instead, the model is almost always tested on interannual timescales for annual plants, where the model prediction that species `store' environmentally favorable periods (`buffered population growth') occurs through seedset into a seedbank (thus the measure of storage is also a direct measurement of fitness, which is required for parameterizing this model with empirical data). 


% END of sent to Elsa on last round of review (20 March 2024):
