Current coexistence models integrating phenology focus almost entirely on phenology as it relates to resource use and growth. While this makes sense from the lens of coexistence theory, which has long been focused on competition (citeMcPeek), it ignores reproductive events, which may fundamentally constrain growth events. 

These combined constraints create a geometric reality to phenology that is generally ignored in most phenology research today. 

... in a way that is constrained, complex, and rarely discussed in most phenological research today.  

The need to grow and reproduce within these bounds ultimately underlies all phenology, but the complexity of the sequence---including the series of events and how each relates to a potential suite of trade-offs---is rarely discussed. 

Within coexistence and more generally in phenological research, much work focuses on one event---erasing the complexity of phenological events that are all generally dependent upon one another. This is perhaps not surprising, as making predictions for sequences of events requires bridging from coexistence to life history theory, which has worked to predict the optimal schedule of growth and reproductive timing across an organism's lifetime. But recent work showing that differences in life history can lead to long-term persistence of species in an otherwise neutral model (where species are effectively equalized, CITE) suggests the lines between coexistence theory and life history theory perhaps could be usefully blurred. 


\subsection*{Future directions}

... how relevant this phenology x competitive ability  trade-off is in natural systems is debatable. 
 
 Advancing our understanding of how phenology fits within coexistence will likely involve a more head-on reckoning with the challenge of time in community assembly models ... STIFF equations here? 

% -ignoring an additionally suite of likely relevant mechanisms that are fluctuation dependent: relative non-linearity of competition and the storage effect. ... seem highly relevant to phenology and could yield new insights. 

Relevant fields overly focused on different plant life histories: coexistence focused on annuals and storage through seedbanks, while much phenology work focuses on trees and their recurring annual events. 

% Refs I need:
John Park
Law 1979
McPeek book