\documentclass[11pt]{article}
\usepackage[top=1.00in, bottom=1.0in, left=1.1in, right=1.1in]{geometry}
\usepackage{graphicx}
\usepackage{natbib}
\usepackage{amsmath}
\usepackage{lineno}
\usepackage{xcolor}
\usepackage{xr-hyper}
\externaldocument{areesms}
\newcommand{\lr}[1]{~\lineref{#1}}
\setlength\parindent{0pt}


\begin{document}
\setlength{\parindent}{0cm}
\setlength{\parskip}{5pt}

Editor comments (we provide below the full context of each review) are in \emph{italics}, while our responses are in regular text. \\ 

{\bf Editor's comments:} \\

\emph{AREES sent me your manuscript to review, and I was very pleased to see how polished it is. It was a pleasure to read.
It is a very clearly written paper, and I enjoyed reading it very much. The goals of the review were clear, as outlined in the Introduction, and it was very clearly structured. I have a few marginal comments, including spotting a few typos, in the manuscript itself. Several of these are only to say, ``nice work!''}\\

Thank you for the positive feedback. It was wonderful to hear the manuscript was easy to read. We believe the comments have greatly improved the manuscript, including sharpening its focus and highlighting current gaps, which are also major opportunities. We have worked to address all comments, and fix all typos. \\

\emph{In general, the review covers a lot of very interesting ground and synthesizes it well. You relate phenology to a lot of existing theory on community assembly and coexistence. Something that doesn’t always come through is the extent to which you are pointing out a) how phenology has already been implicit in these models and is therefore an unacknowledged and under-appreciated contributor to these dynamics, versus b) the extend to which phenology has not been adequately incorporated and that doing so would lead to substantively different and more accurate conclusions.}

\emph{That latter message could be developed and brought out a bit more clearly, possibly with some strategic summary tables or graphics.
Here are some comments to consider, in the order of the text.}

Great point about clarifying whether phenology is implicit/unacknowledged or not incorporated ... across everything we cover in the paper, it is of course a mix, but moreso that phenology has not been adequately incorporated, and thus we may fail to correctly understand or predict systems. We have worked to highlight this more throughout, including lines \lr{notincluded1S}-\lr{notincluded1E} and\lr{notincluded2S}-\lr{notincluded2E}  ... \\

\emph{Phenological assembly, priority effects, and coexistence:\\
The description of phenology influencing filtering of both abiotic and biotic factors was especially clear, and led very logically to the discussion of priority effects, and then mechanisms of coexistence mediated by temporal niche separation. These three sections were very well integrated.}

Thanks.\\

\emph{In the section on ``Phenological coexistence,'' I enjoyed the discussion of phenological niche partitioning. I was intrigued by Fig 2, mentioned (without references) in a previous section. The dynamics of Fig 2 seem pertinent to this section, and it would have been interesting to refer to it in this section, along with a discussion of the references that inspired it.}

Great point, we have added references to this figure now in the `Phenological coexistence' section, specifically line \lr{reffig1}. We have also added references where we introduce Figure 1 (lines \lr{morefsE}-\lr{morefsE}). \textcolor{red}{Not sure how to add much more without making the section even longer ... if you have ideas, go for it! I prioritized explaining the storage effect.)}\\

\emph{Also in that section, I had some difficulty appreciating the concept of “storage,” I think because it clearly was referring to a body of literature and a corresponding concept that is not in my main line of view, and it was not completely explained within the text. If you can refer to the phenomenon itself (either the ability to store resources over time, or more pertinent to phenology, the ability to exhibit dormancy or quiescence of specific life stages) rather than referring to it as an abstract concept of “storage of favorable environments” I think it would have been easier to follow. I understand that you want to make contact with the pre-existing theoretical concepts, but the actual role of phenology gets a bit obscured. Related to this, Fig 1 could emphasize the unique consequence of “storage” rather than showing how phenology can be considered a form of storage.}

This is a a great point that we have addressed through adding a paragraph more explicitly explaining the storage effect and then extending other surrounding paragraphs to make sure the explanation is clear. See lines \lr{definestorage1} and \lr{definestorage2S}-\lr{definestorage2E} and also lines \lr{definestorage3S}-\lr{definestorage3E} and lines \lr{definestorage4S}-\lr{definestorage4E}. We also adjusted the `storage' related figure and its caption.\\


\emph{I really liked that you discussed correlations between phenology and other traits, and how you related that to tradeoffs and existing durable theories of coexistence. You describe very clearly how phenology is implicitly or explicitly incorporated in these theories. What is less obvious is what new predictions/outcomes are expected by explicitly incorporating stage-specific phenology (which you emphasize, as opposed to a more general phenological descriptor of the whole life cycle). Can you describe cases (or even highlight one case study) that shows that predictions or outcomes regarding whether a specific species can successfully colonize and persist in a community depended on it a stage-specific phenological trait (as opposed to general life-history strategy)? That would strengthen your emphasis on stage-specific phenology over less defined phenological categories (p.11).}

\textcolor{red}{Could possibly also cite here new text on lines \lr{definestorage4S}-\lr{definestorage4E}?}\\

\emph{I also saw a clear connection between these correlations and the filtering dynamics that you discussed earlier. That is, phenology and tolerances to particular environmental stressors (abiotic and biotic) can evolve to be associated. How would that association alter the filtering dynamics that are so fundamental to community assembly?}

\textcolor{red}{Err ... I will try to edit last figure and maybe we can sneak something into the caption?}\\


\emph{Other potential connections:\\
I wonder whether some mention of how phenology-mediated assembly processes might influence the phylogenetic or functional diversity of communities could be appropriate. You pay a lot of attention to the association between phenology and other traits that influence assembly and coexistence, and a lot of attention to the potential (evolved) syndromes of associations between life histories, phenology, and other traits. Does variation in phenology strengthen or loosen these associations? If phenology presents another axis of variation, can variation in phenology allow different combinations of traits that influence these important tradeoffs—more than if phenology were not a factor? Or does phenology contribute to the establishment of those tradeoffs, canalizing variation into a few syndromes? Would there be more, or less, functional diversity (variation in physiological tolerances, competitive ability, etc) if phenology varied versus did not vary?}

\textcolor{red}{}\\

\emph{In the Intro, you mentioned that phenologically-mediated community assembly and coexistence dynamics pertain to species introductions and invasions. While you do discuss that within the different sections, I suspect that having some figure or text box that consolidates it as a distinct message would be of interest to the readership. Likewise, a similar summary of the pertinence of these dynamics to restoration could also be interesting. These are just suggestions.}

\textcolor{red}{Needs help...}\\

\emph{It would help to have some sort of table or figure that summarizes some of the main points about how phenology alters community assembly, and how considering it explicitly leads to different conclusions from existing models that only implicitly include it. Alternatively, if one of the main points is that these important models do already include phenology, then point out the significance of acknowledging that explicitly—what additional important insight is gained or application made possible (I liked the reference to restoration. If you can find a way to make more explicit what unresolved questions become newly resolved by incorporating phenology in theoretical and empirical analyses—and summarize that in a concise graphic or table, I’m sure that figure or table would be highly cited!}

\textcolor{red}{Needs help...}\\

\emph{This was such an interesting and smooth read. The suggestions here are meant to help sharpen the presentation of how considering phenology substantively alters predicted outcomes related to community assembly and biodiversity maintenance, and how such insight
can be applied. There extent reflects how engaging the topic is to me, and the skill with which you both discuss it. comments were helpful.}

Thanks!\\

{\bf Editor's comments in PDF:} 

In regards to our definition of phenology: \emph{I suppose I am not familiar with the narrower definition(s) that you are referring to, so I'm not sure what the refinement is... }

We explain this in lines \lr{defineS}-\lr{defineE}:
\begin{quote}
This definition is intentionally more inclusive than some other definitions, which focus on recurring or seasonal events, and thus can narrow phenology to only certain plant types or biomes (e.g., woody species in the temperate zone). We find this narrowing artificial and think it can limit a broader understanding the selective pressures on phenology that---as we outline below---are critical for understanding the role of phenology in community assembly. Our definition thus includes both leafout of trees and seed germination of annual plants, at the same time that it encompasses fruiting, flowering and transitions in and out of these phases, such as dormancy and vernalization.
\end{quote}


Reference to \citet{godoy2014} \emph{Does this include temporal niche partitioning?  Or does temporal separation increase other aspects of niche differentiation (such as environmental tolerances)?  If the former, you could add `spatio-temporal niche difference' or something to indicate that temporal niche differentiation is included.}

This is a great question with a complicated answer. The exact `niche differences' \citet{godoy2014} refer to here come from a parameterized competition model that specifically does not make it easy to link `niche differences' to a mechanism. This problem has recently been explained in  \citet{mcpeek2022coexistence} and we now reference this while addressing comment parenthetically on lines \lr{whatnicheS}-\lr{whatnicheE}.

\bibliography{areesbib.bib}

\end{document}

